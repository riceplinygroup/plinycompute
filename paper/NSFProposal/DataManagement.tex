
\begin{center}{\textbf{Data Management Plan}}\end{center}

$$P(\theta | \Psi) = \prod_i \textrm{Normal}(f_{t(\theta_i)}(\theta_i) | \Psi, \sigma_{t(\theta_i)})$$

$$f$$

$$P(\Psi | \theta)$$ 

$$P(F | \Psi)$$

The proposed project will develop software implementing PLinyCompute, as described in
this proposal. The software and testbeds will be developed, managed, distributed, and supported as a collaborative software
project, with a view towards an open source release when the research has been completed. Components of the system
will be distributed under GPL, BSD and EPL licenses, depending on the provenance of any external open source
code used. All code will be preserved and maintained for a minimum of three years beyond the award period
as required by NSF guidelines. Based on our experience with past projects, we expect the code to be preserved for
a much longer duration.
All electronic data will be preserved in multiple on-site backups in the form of DVDs and RAID hard drive
storage. Copies of the electronic data will be preserved off-site at Rice University's storage facilities. If requested,
access to the data will be provided via contact with the PI. Data will be available for access and sharing as soon as
is reasonably possible, and no later than two years after its acquisition. The data will be preserved for at least three
years beyond the award period, as required by NSF guidelines.
We do not anticipate that there will be any significant intellectual property issues involved with the acquisition of
the data. In the event that discoveries or inventions are made in direct connection with this data, access to the data
will be granted upon request once appropriate invention disclosures and/or provisional patent filings are made.
The data acquired and preserved in the context of this proposal will be further governed by Rice University's
policies pertaining to intellectual property, record retention, and data management.

