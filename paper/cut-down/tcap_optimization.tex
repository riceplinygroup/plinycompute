
\section{Optimizing TCAP}
\label{sec:optimizer}

Optimizability is one of the drivers for
the decision to compile all computations expressed in PC's lambda calculus into TCAP.  
TCAP resembles relational algebra, and it is similarly amenable to rule- and cost-based optimization
using a combination of methods from relational query optimization and classical compiler construction.

PC's optimizer is currently implemented
in Prolog; a series of transformations are fired iteratively to improve the plan until the plan cannot be improved further.
For an example of the sort of optimization present in PC, consider the task of removing redundant method calls.  Imagine that a user
supplies a \texttt{SelectionComp} with the following \texttt{getSelection ()}:

\begin{codesmall} 
Lambda <bool> getSelection (Handle  <Emp> emp) {
   return makeLambdaFromMethod
      (emp, getSalary) > 50000 &&
      makeLambdaFromMethod (emp, getSalary) < 10000;
}	
\end{codesmall}

\noindent PC would compile this into the following TCAP:

\begin{codesmall}
JK2_1(emp,mt1) <= APPLY(In(emp), In(emp), 'Sel_43',
'method_call_1',[('type', 'methodCall'),('methodName',
'getSalary')]);

JK2_2(emp,bl1) <= APPLY(JK2_1(mt1), JK2_1(emp), 
'Sel_43','>_1',[('type','const_comparison'),('op','>')]);

JK2_3(emp,bl1,mt2) <= APPLY(JK2_2(emp), JK2_2(emp,bl1), 
'Sel_v3', 'method_call_2',[('type','methodCall'), 
('methodName', 'getSalary')]);

JK2_4(emp,bl1,bl2) <= APPLY(JK2_3(mt2), JK2_3(emp,bl1), 
'Sel_43','<_1',[('type','const comparison'),('op','<')]);

JK2_5(emp,bl3) <= APPLY(JK2_4(bl1,bl2), JK2_4(emp), 
'Sel_43', '&&_1',[('type','bool_and')]);

JK2_6(emp) <= FILTER(JK2_5(bl3), JK2_5(emp), 'Sel_43',[]);
\end{codesmall}

\noindent
This TCAP program first calls the method \texttt{getSalary ()} on \texttt{In::emp} to produce a new vector list \texttt{JK2\_2}, storing the result
of the method call in \texttt{JK2\_1.mt1}.  After comparing \texttt{JK2\_2.bl1} to \texttt{50000}, the result of the method call is dropped.
The method is then called once again on \texttt{JK2\_2.emp} and the result compared with \texttt{100000} to produce \texttt{JK2\_4}, at which 
point the two boolean vectors are ``anded'' and the result is filtered.

Obviously, there is a redundancy here as the method \texttt{getSalary ()} will be called twice.
If \texttt{getSalary} simply accesses a data member, the additional call is costless.  But in the general case a method call may run an arbitrary
computation.  Hence, the second call should automatically be removed as being redundant 
(by definition, all method calls evaluated during computation should
be purely functional, and so they must return the same value when called a second time).
The TCAP optimization rule leading to its removal is:

\vspace{5pt}

\noindent
(1) If two \texttt{APPLY} operations are both of type \texttt{methodCall} and both invoke the same \texttt{methodName};

\noindent
(2) And one \texttt{APPLY} operation is the ancestor of the other in the TCAP graph;

\noindent
(3) And both \texttt{APPLY} operations operate over the same data object;

\noindent
(4) Then the second \texttt{APPLY} operation can be removed, and the
result of the first \texttt{APPLY} carried through the graph.

\vspace{5pt}

\noindent
In our example, the
optimized
TCAP program is:

\begin{codesmall}
JK2_1(emp,mt1) <= APPLY(In(emp), In(emp),'Sel_43', 
'method_call_1',[('type','methodCall'),('methodName',
'getSalary')]);

JK2_2(emp,mt1,bl1) <= APPLY(JK2_1(mt1),JK2_1(emp,mt1), 
'Sel_43','>_1',[('type','const comparison'),('op','>')]);

JK2_4(emp,bl1,bl2) <= APPLY(JK2_3(mt1),JK2_3(emp,bl1),
'Sel_43','<_1',[('type','const comparison'),('op','<')]);

JK2_5(emp,bl3) <= APPLY(JK2_4(bl1,bl2), JK2_4(emp), 
'Sel_43','&&_1',[('type', 'bool_and')]);

JK2_6(emp) <= FILTER(JK2_5(bl3), JK2_5(emp),'Sel_43',[]);
  
\end{codesmall}

\noindent
Currently, the optimizations implemented in PC are rule-based (such as pushing down selections).  We plan to work on cost-based optimization
in the future, which is a challenging research problem because of lack of statistics over the arbitrary PC \texttt{Object}s.
